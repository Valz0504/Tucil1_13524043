\chapter{Implementasi}

Queens adalah gim logika yang tersedia pada situs jejaring profesional LinkedIn. Tujuan dari gim ini menempatkan \textit{queen} pada sebuah papan persegi berwarna sehingga terdapat hanya satu \textit{queen} pada tiap baris, kolom, dan daerah warna. Selain itu, satu \textit{queen} tidak dapat ditempatkan bersebelahan dengan \textit{queen} lainnya, termasuk secara diagonal. Penyelesaian gim ini diprogram dengan bahasa Java, memanfaatkan \textit{toolkit} JavaFX untuk GUI (Graphical User Interface)-nya. Algoritma \textit{brute force} ditekankan untuk menyelesaikan gim ini.

\section{Brute Force}

Algoritma \textit{brute force} yang diimplementasikan bersifat murni, yaitu mengenumerasi seluruh penempatan \textit{queen} yang mungkin berdasarkan aturan permainan. Penulis memanfaatkan \textit{backtracking} tanpa adanya \textit{pruning} sehingga algoritma tetap mengecek seluruh kemungkinan posisi \textit{queen}. 

Algoritma utama \textit{brute force} terdapat di prosedur \texttt{solveBruteForce(int row, Board papan)} yang alur kerja sederhananya adalah sebagai berikut:

\begin{enumerate}
    \item Pada awal pemanggilan prosedur (pertama kali), parameter yang di-\textit{assign} adalah $0$ dan \texttt{papan} yang sudah terisi.
    \item Algoritma akan mencoba meletakkan \textit{queen} di setiap kolom dengan adanya \textit{backtracking} di baris \texttt{row}, lalu melanjutkan pemanggilan dengan parameter \texttt{row} yang lebih dalam.
    \item Jika pada saat ini nilai \texttt{row} sama dengan baris papan yang sebenarnya, akan dilakukan pengecekan apakah konfigurasi papan yang sudah ditempatkan \textit{queen} valid.
    \item Jika valid, atribut \texttt{solution[][]} akan di-\textit{update} dengan konfigurasi papan tersebut.
\end{enumerate}

Dikarenakan penulis memberikan opsi untuk mengoptimasi komputasi dengan menyediakan fitur \textit{optimized} maka terdapat trik yang dapat mempercepat komputasi, yaitu dengan memoisasi kolom dan warna yang sudah terpakai, kasus yang bertabrakan dengan hal tersebut akan dilewati dan tidak dicek.

Algoritma murni tanpa optimasi memiliki kompleksitas waktu $O(N^N)$ karena untuk setiap baris, akan dicoba $N$ kolom dari konfigurasi papan sebelumnya.

\begin{algorithm}[H]
\caption{Algoritma Brute Force Queen}
\begin{algorithmic}[1]
\Procedure{solveBruteForce}{row, Papan}

\If{cancelled}
    \State \Return
\EndIf

\If{isOptimized = true}
    \State caseCount $\gets$ caseCount + 1
\EndIf

\If{row = Papan.getRow()}
    \If{isOptimized = true}
        \State solutionCount $\gets$ solutionCount + 1
        \For{$i \gets 0$ to Papan.getRow() - 1}
            \For{$j \gets 0$ to Papan.getCol() - 1}
                \State solution[i][j] $\gets$ board[i][j]
            \EndFor
        \EndFor
        \State \Return
    \Else
        \State caseCount $\gets$ caseCount + 1
        \If{isBoardValid(Papan.getRow() - 1, Papan)}
            \State solutionCount $\gets$ solutionCount + 1
            \For{$i \gets 0$ to Papan.getRow() - 1}
                \For{$j \gets 0$ to Papan.getCol() - 1}
                    \State solution[i][j] $\gets$ board[i][j]
                \EndFor
            \EndFor
        \EndIf
        \State \Return
    \EndIf
\EndIf

\For{$col \gets 0$ to Papan.getCol() - 1}
    \If{isOptimized = true}
        \State warna $\gets$ Papan.getElmt(row, col) - 'A'
        \If{usedColumn[col] = 0 \textbf{and} color[warna] = 0 \textbf{and} check8Direction(row, col, Papan)}
            \State color[warna] $\gets$ 1
            \State board[row][col] $\gets$ 1
            \State usedColumn[col] $\gets$ 1

            \State debugCases(Papan)
            \State solveBruteForce(row + 1, Papan)

            \State color[warna] $\gets$ 0
            \State board[row][col] $\gets$ 0
            \State usedColumn[col] $\gets$ 0
        \EndIf
    \Else
        \State board[row][col] $\gets$ 1
        \State debugCases(Papan)
        \State solveBruteForce(row + 1, Papan)
        \State board[row][col] $\gets$ 0
    \EndIf
\EndFor

\EndProcedure
\end{algorithmic}
\end{algorithm}

\section{Fitur Pelengkap}

Program juga mengimplementasikan beberapa fitur lainnya, antara lain:

\begin{enumerate}
    \item Graphical User Interface (GUI) : Antarmuka grafis menggunakan JavaFX
    \item Live Update : Saat solver berjalan, papan di-\textit{render} ulang setiap langkah dengan delay yang bisa diatur menggunakan slider sehingga pengguna dapat melihat proses pencarian secara \textit{real-time}
    \item Output as Text : Solusi bisa disimpan ke file \texttt{.txt} 
    \item Output as Image : Solusi bisa disimpan sebagai file \texttt{.png}
\end{enumerate}

