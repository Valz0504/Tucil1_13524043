\chapter{Lampiran}
\url{https://github.com/Valz0504/Tucil1_13524043}

\begin{table}[H]
    \centering
    \caption{Evaluasi}
    \label{tab:checklist_eval}
    \renewcommand{\arraystretch}{1.5}
    \begin{tabular}{|c|p{10cm}|c|c|}
        \hline
        \textbf{No} & \centering \textbf{Poin} & \textbf{Ya} & \textbf{Tidak} \\ \hline
        1 & Program berhasil di kompilasi tanpa kesalahan & \checkmark & \\ \hline
        2 & Program berhasil di jalankan & \checkmark & \\ \hline
        3 & Solusi yang diberikan program benar dan mematuhi aturan permainan & \checkmark & \\ \hline
        4 & Program dapat membaca masukan berkas .txt serta menyimpan solusi dalam berkas .txt & \checkmark & \\ \hline
        5 & Program memiliki Graphical User Interface (GUI) & \checkmark & \\ \hline
        6 & Program dapat menyimpan solusi dalam bentuk file gambar & \checkmark & \\ \hline
        7 & Program dapat membaca masukan gambar & & \checkmark \\ \hline
    \end{tabular}
\end{table}

Tugas ini disusun sepenuhnya tanpa bantuan kecerdasan buatan (\textit{Generative AI}),
melainkan hasil pemikiran dan analisis mandiri.
\begin{flushright}
    \begin{minipage}{0.4\textwidth}
        \centering \includegraphics[width=4cm]{img/ttd.jpg}\\
        Emilio Justin
    \end{minipage}
\end{flushright}