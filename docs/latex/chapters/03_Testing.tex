\chapter{Pengujian}

Penulis menyarankan apabila penguji menggunakan program dan memasukkan input dengan ukuran yang cukup besar, Window GUI program dapat di-\textit{resize} secara menual (tidak responsive).

\section{Test 1: Papan berukuran $9 \times 9$}

\begin{table}[H]
    \centering
    \begin{minipage}{0.45\textwidth}
        \centering
        \captionof{table}{Input Test 1}
        \renewcommand{\arraystretch}{1.2}
        \setlength{\tabcolsep}{40pt}
        \begin{tabular}{|l|}
            \hline
            \phantom{text}\\
            AAABBCCCD\\
            ABBBBCECD\\
            ABBBDCECD\\
            AAABDCCCD\\
            BBBBDDDDD\\
            FGGGDDHDD\\
            FGIGDDHDD\\
            FGIGDDHDD\\
            FGGGDDHHH\\
            \phantom{text}\\
            \hline
        \end{tabular}
    \end{minipage}
\end{table}

\subsubsection*{Output sebagai \texttt{.txt}}
\begin{lstlisting}[basicstyle=\ttfamily\small, caption="Test 1 Output"]
AAABBCC#D
ABBB#CECD
ABBBDC#CD
A#ABDCCCD
BBBBD#DDD
FGG#DDHDD
#GIGDDHDD
FG#GDDHDD
FGGGDDHH#
\end{lstlisting}

\subsubsection*{Output sebagai gambar}
\begin{figure}[H]
    \centering
    \includegraphics[width=10cm]{img/testing/tc1.png}
    \caption{Output GUI --- Test 1}
\end{figure}

\section{Test 2: Papan berukuran $8 \times 8$}

\begin{table}[H]
    \centering
    \begin{minipage}{0.45\textwidth}
        \centering
        \captionof{table}{Input Test 2}
        \renewcommand{\arraystretch}{1.2}
        \setlength{\tabcolsep}{40pt}
        \begin{tabular}{|l|}
            \hline
            \phantom{text}\\
                AAAAAAAA\\
                BCCCDDDA\\
                BCEEEEDA\\
                BCCEEFFA\\
                BBFEEFAA\\
                BBFGGFAA\\
                HHFFFFAA\\
                HAAAAAAA\\
            \phantom{text}\\
            \hline
        \end{tabular}
    \end{minipage}
\end{table}

\subsubsection*{Output sebagai \texttt{.txt}}
\begin{lstlisting}[basicstyle=\ttfamily\small, caption="Test Case 2 Output"]
AAAAAAA#
BC#CDDDA
BCEEEE#A
BCCE#FFA
B#FEEFAA
BBF#GFAA
HHFFF#AA
#AAAAAAA
\end{lstlisting}

\subsubsection*{Output sebagai gambar}
\begin{figure}[H]
    \centering
    \includegraphics[width=10cm]{img/testing/tc2.png}
    \caption{Output GUI --- Test 2}
\end{figure}

\section{Test 3: Papan berukuran $7 \times 7$}

\begin{table}[H]
    \centering
    \begin{minipage}{0.45\textwidth}
        \centering
        \captionof{table}{Input Test 3}
        \renewcommand{\arraystretch}{1.2}
        \setlength{\tabcolsep}{40pt}
        \begin{tabular}{|l|}
            \hline
            \phantom{text}\\
                ABBCCDD\\
                AABBCDD\\
                ABBBBEE\\
                BBBBBBE\\
                BBFBBBB\\
                BFFFBBB\\
                GGGGGBB\\
            \phantom{text}\\
            \hline
        \end{tabular}
    \end{minipage}
\end{table}

\subsubsection*{Output sebagai \texttt{.txt}}
\begin{lstlisting}[basicstyle=\ttfamily\small, caption="Test Case 3 Output"]
ABB#CDD
AABBC#D
#BBBBEE
BBBBBB#
BB#BBBB
BFFF#BB
G#GGGBB
\end{lstlisting}

\subsubsection*{Output sebagai gambar}
\begin{figure}[H]
    \centering
    \includegraphics[width=10cm]{img/testing/tc3.png}
    \caption{Output GUI --- Test 3}
\end{figure}

\section{Test 4: Papan tidak memiliki solusi}

\begin{table}[H]
    \centering
    \begin{minipage}{0.45\textwidth}
        \centering
        \captionof{table}{Input Test 4}
        \renewcommand{\arraystretch}{1.2}
        \setlength{\tabcolsep}{40pt}
        \begin{tabular}{|l|}
            \hline
            \phantom{text}\\
                AAAAAA\\
                AAAAAA\\
                AAAAAA\\
                AAAAAA\\
                AAAAAA\\
                ABCDEF\\
            \phantom{text}\\
            \hline
        \end{tabular}
    \end{minipage}
\end{table}

\subsubsection*{Output sebagai gambar}
\begin{figure}[H]
    \centering
    \includegraphics[width=10cm]{img/testing/tc4.png}
    \caption{Output GUI --- Test 4}
\end{figure}

\section{Test 5: Input tidak valid}

\begin{table}[H]
    \centering
    \begin{minipage}{0.45\textwidth}
        \centering
        \captionof{table}{Input Test 5}
        \renewcommand{\arraystretch}{1.2}
        \setlength{\tabcolsep}{40pt}
        \begin{tabular}{|l|}
            \hline
            \phantom{text}\\
                ABBCCDD\\
                AABBCDD\\
                ABBBBEE\\
                BBBBBBE\\
                BBFBBBB\\
                BFFFBBB\\
                GGGG\\
            \phantom{text}\\
            \hline
        \end{tabular}
    \end{minipage}
\end{table}

\subsubsection*{Output sebagai gambar}
\begin{figure}[H]
    \centering
    \includegraphics[width=10cm]{img/testing/tc5.png}
    \caption{Output GUI --- Test 5}
\end{figure}

\section{Test 6: Papan tidak memiliki solusi}

\begin{table}[H]
    \centering
    \begin{minipage}{0.45\textwidth}
        \centering
        \captionof{table}{Input Test 6}
        \renewcommand{\arraystretch}{1.2}
        \setlength{\tabcolsep}{40pt}
        \begin{tabular}{|l|}
            \hline
            \phantom{text}\\
                AAABB\\
                ACBBB\\
                ACDBB\\
                ACDDB\\
                CCDDB\\
            \phantom{text}\\
            \hline
        \end{tabular}
    \end{minipage}
\end{table}

\subsubsection*{Output sebagai gambar}
\begin{figure}[H]
    \centering
    \includegraphics[width=10cm]{img/testing/tc6.png}
    \caption{Output GUI --- Test 6}
\end{figure}

\section{Test 7: Papan berukuran $5 \times 5$}

\begin{table}[H]
    \centering
    \begin{minipage}{0.45\textwidth}
        \centering
        \captionof{table}{Input Test 7}
        \renewcommand{\arraystretch}{1.2}
        \setlength{\tabcolsep}{40pt}
        \begin{tabular}{|l|}
            \hline
            \phantom{text}\\
                AAABBC\\
                ADDBBC\\
                ADDEEC\\
                AFFEEC\\
                AFFEEC\\
                AFFGGC\\
            \phantom{text}\\
            \hline
        \end{tabular}
    \end{minipage}
\end{table}

\subsubsection*{Output sebagai \texttt{.txt}}
\begin{lstlisting}[basicstyle=\ttfamily\small, caption="Test Case 7 Output"]
AAABB#
ADD#BC
A#DEEC
AFFE#C
AF#EEC
#FFGGC
\end{lstlisting}

\subsubsection*{Output sebagai gambar}
\begin{figure}[H]
    \centering
    \includegraphics[width=10cm]{img/testing/tc7.png}
    \caption{Output GUI --- Test 7}
\end{figure}